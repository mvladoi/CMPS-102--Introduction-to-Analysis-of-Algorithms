\documentclass[11 pt]{article}
\usepackage{fullpage,amsthm,amsfonts,amssymb,epsfig,amsmath,times,amsthm}
\usepackage{algorithm,algpseudocode}
\usepackage{setspace}

\newtheorem{theorem}{Theorem}
\newtheorem{claim}[theorem]{Claim}




\title{ CMPS 102 --- Quarter  Spring 2017 --  Homework 2}
\author{VLADOI MARIAN}
\date{\today}

\begin{document}
\maketitle



\begin{center}
{\bf I have read and agree to the collaboration policy.Vladoi Marian}\\
{\bf Name of students I worked with: Victor Shahbazian and Mitchell Etzel }
\end{center}


\section*{Solution to Problem 1: Divide and Conquer}		



\begin{proof}
1A. We should consider  the Walsh Hadamard matrix of order n, named H(n).\\
 


 
$
\begin{bmatrix}
h_{00} & h_{01} & h_{02} $ ... $ h_{0n} \\
h_{10} & h_{11} & h_{12} $ ... $ h_{1n} \\
.........\\
h_{n0} & h_{n1} & h_{n2} $ ... $ h_{nn} 
\end{bmatrix}
$

Were $h_{i,j} \forall \ i,j \ \in \  {0, \ n-1 }$ are elements of the matrix.
These elements can be found as follow :\\
For  $\ i, j \geq  0$, we write the binary numbers  of i,j as : \\
$ \sum_{x= 0}^{n-1} i_x2^x   \   and \  \sum_{x= 0}^{n-1} j_x2^x  $;
Where $i_x \in \{0,1\} \forall \{0,1, .... n-1\}$;
And $j_x \in \{0,1\} \forall i \{0,1, .... n-1\}$.\\
This implies that $ h{i,j} = (-1)^{\sum_{x = 0}^ {n-1} \oplus i_x, j_x }$ where $ \oplus $ denotes adition modulo $-2$. \\ The idea is that we use the base 2 of the indeces i,j to create the dot product. \\
The smallest Walsh Hadamard is H(0). Using the previous formula for h(0,0) we get the matrix [1].
Every time we increase a Walsh hadamard matrix H(n) to H(n+1), the final matrix wil be a square matrix of size $2^{n+1}$.H(n) will always be the top left matrix in H(n+1). We want to show that any position h(i,j) in this H(n) matrix = $h(i, j+2^{n-1})$ in H(n+1) matrix =  $h(i+2^{n-1}, j)$ in H(n+1) matrix = $ - h(i+2^{n-1}, j+2^{n-1})$ in H(n+1) matrix.   Acording to our dot product of indices (in binary) we will always have this case.  
\end{proof}




\begin{proof}
1.B Proof by induction:\\
Base case: for n = 1 we have :
\[
H_1 = \frac{1}{\sqrt[]{2^1}}
\begin{bmatrix}
1 & 1\\
1 & -1
\end{bmatrix}
= 
\frac{1}{\sqrt[]{2^1}} \sqrt[]{\sum_{i=1} ^{2^1} i^2 }
= 1
\]

Assume for any $k>1$ that : 
\[
H_1 = \frac{1}{\sqrt[]{2^k}}
\begin{bmatrix}
H_{k-1} & H_{k-1}\\
H_{k-1} & H_{k-1}
\end{bmatrix}
= 
\frac{1}{\sqrt[]{2^k}} \sqrt[]{\sum_{i=1} ^{2^k} i^2 }
= 
\frac{1}{\sqrt[]{2^k}} \sqrt[]{2^k}
=1
\]

We must show that the statement is true for k+1:
\[
H_1 = \frac{1}{\sqrt[]{2^k}}
\begin{bmatrix}
H_{k} & H_{k}\\
H_{k} & -H_{k}
\end{bmatrix}
= 
\frac{1}{\sqrt[]{2^{k+1}}} \sqrt[]{\sum_{i=1} ^{2^{k+1}} i^2 }
=   
\frac{1}{\sqrt[]{2^{k}}} \sqrt[]{\sum_{i=1} ^{2^{k}} i^2 } (By \ Induction \  Hypothesis = 1)  \frac{1}{\sqrt[]{2^{k+1}}} \sqrt[]{2^{k+1} } = 1
\]

The Eucledian norm of every column and every row = 1 $\forall n$
\end{proof}

\begin{proof}

1.C We proved on Claim 2 that The Eucledian norm of every column and every row of $H_n$ is 1. Because we are dealing with Walsh-Hadamard matrix, the same aproached can be used to prove that every column of $H_n$ has Euclidean norm of 1.\\
We will prove next, that the dot product of any two columns of $H_n$ equals to 0.\\
Our $H_n$ matrix, whose columns and rows are indexed by vectors x, y $\in \{0,1\}^n $, has entries: $$H_n(y,x) = (-1)^{\sum_{i}^{\infty}x_iy_i}$$
Suppose that x and z are two different columns, $x_j \ne z_j$. The inner product between the two columns is :
$$   \sum_{y=1}^{\infty}H(y,x) H(y,z) = \sum_{y=1}^ {\infty} (-1)^{\sum_{i=1}^{\infty}}y_i(x_i+z_i)$$
Next we decompose y in $y_j, y_{-j}$, wher $y_{-j} \in \{0,1\}^{n-1}$ The inner product becomes :
$$  \sum_{y=1}^{\infty}H(y,x) H(y,z) =   \sum_{y_j,y_{-j}} (-1)^{y_j}(-1)^{\sum_{i \ne j}y_i(x_i+z_i) } = \sum_{y_{-j}}(-1)^{\sum_{i \ne j}y_i(x_i+z_i) } \sum_{y_j= 0}^{1}(-1)^{y_j} = 0$$ 
\end{proof}


\begin{proof}

1.D We will use Divide and Conquer aproach.\\
Supose we have a $H_n\ matrix$. This is a $2^n  \ by \ 2^n$ matrix. We know that this $H_n$ can be computed in terms of $H_{n-1}$. Then the column vector of this matrix, called v has size n. We want to express this vector in terms of two vectors of size $=\frac{n}{2} = 2^{n-1}$. Lets name the upper part of this vector $v_x$ and the lower part of this vector $v_y$ . So each iteration, the size of our problem decreases.We will use the recursion until we hit the base case. Then we will start to construct our solution of the algorithm. 
\[
H_n v = 
\begin{bmatrix}
H_{n-1} & H_{n-1}\\
H_{n-1} & -H_{n-1}
\end{bmatrix}
\begin{bmatrix}
v_x\\
v_y
\end{bmatrix}
= 
\begin{bmatrix}
H_{n-1}v_x & H_{n-1}v_y\\
H_{n-1}v_x & -H_{n-1}v_y
\end{bmatrix}
\]


\[
H_{n-1} v = 
\begin{bmatrix}
H_{n-2} & H_{n-2}\\
H_{n-2} & -H_{n-2}
\end{bmatrix}
\begin{bmatrix}
v_{x_x}\\
v_{y_y}
\end{bmatrix}
= 
\begin{bmatrix}
H_{n-2}v_{x_x} & H_{n-1}v_{y_y}\\
H_{n-1}v_{x_x} & -H_{n-1}v_{y_y}
\end{bmatrix}
\]

The algorithm for this problem.\\

MatrixVector($Hn, v$)\\
1. If $H_0$ return v.\\
2. vector $v_x$ = the upper part of the vector v, having size $2^{n-1}$\\
3. vector $v_y$ = the lower part of the vector v, having the same size\\
4. vector $t_1$ = MatrixVector($H_{n-1} v_x$)\\
5. vector $t_2$ = MatrixVector ($H_{n-1}v_y$)\\
6. vector t = $[t_1 + t_2, t_1 + (-t_2) ]$ we are constructing the solution vector from the base case\\
7. Return t \\

Assume that T(n) is the cost of solving $H_n v$. Then the cost of solving $H_{n-1}v_x,  \ and \ H_{n-1}v_y \ is  \ T(\frac{n}{2})$. The cost of solving the base case, making the aditions , is O(n). The recurrence relation is $T(n) = 2T( \ \frac{n}{2}) \ + \ O(n)$. Solving this with Master Theorem , case number 2, we get the running time O(n log(n)).
\end{proof}


\end{document}
