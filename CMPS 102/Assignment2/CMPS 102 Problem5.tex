\documentclass[11 pt]{article}
\usepackage{fullpage,amsthm,amsfonts,amssymb,epsfig,amsmath,times,amsthm}
\usepackage{algorithm,algpseudocode}
\usepackage{setspace}

\newtheorem{theorem}{Theorem}
\newtheorem{claim}[theorem]{Claim}




\title{ CMPS 102 --- Quarter  Spring 2017 --  Homework 2}
\author{VLADOI MARIAN}
\date{\today}

\begin{document}
\maketitle



\begin{center}
{\bf I have read and agree to the collaboration policy.Vladoi Marian}\\
{\bf Colaborators None }
\end{center}


\section*{Solution to Problem 5: Optional}

The temperatures ranges from 1 to n degrees. I choose the temperature s . He is wiling to let the temperature at some t temperature or any lower temperature.\\

\textbf{a. Change my mind exactly once.}\\
The strategy  I use.\\

1. Choose s to be n . \\
2. Set integer i = 0;\\
3. While (s ==  too warm) \\
  \{\\
  4. chose s to be =   $n - 2^i $.   (choose  $n- 2^0,  n - 2^1, n-2^2, ...$)\\
  5.  increase i by one.  \\
 ( At the end of the while loop I know that t has to be in the interval $ n - 2^i \  to \  n- 2^{i-1}$)\\
 (I can find this interval in logn choices)

\}\\
6. Now that I know the interval. I change my mind. \\
7. I start againg from $2^{i-1} $ to choose, one by one value , until  I would go lower than t and he would agree with s. \\
( the last round of negociations is constant, and I claim that I finish the process  after log(n) negociations )\\


\textbf{ b. Change my mind k times , where $k > 1$.}\\
The strategy I use.\\

1. Choose s to be n . \\
2. Set integer i = 0;\\
3. While ($k > 1$)\{\\
4. While (s ==  too warm) 
  \{\\
  5. chose s to be =   $n - k^i $.   (choose  $n- k^0,  n - k^1, n-k^2, ...$)\\
  6.  increase i by one.  \\
 ( At the end of the while loop I know that t has to be in the interval $ n - k^i \  to \  n- k^{i-1}$)\\
\}\\
7. Now that I know the interval, t has  to be in the interval $n-k^i \ to \ n-k^{i-1}$ , I will change my mind. \\
8. Repeat the outer while loop with k - 1 and looking in the interval $ n-k^i \ to \ n-k^{i-1}$. This means that now n = $n-k^{i-1}$\\
\}\\
7.When k = 1 I exit the loops,  and  I  found the range where t is. In  constant time I cheeck one by one solution in this range. \\
(Each round of negociation $f_k(n)$  takes $log_k(n)$ negociations, for $k > 1$ .)\\
(The total negociations = $\sum_{i=1}^{k}  log_i (n)  \ + \ constant \ time \ when\  i = 1.$


\end{document}
