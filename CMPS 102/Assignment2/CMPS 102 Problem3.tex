\documentclass[11 pt]{article}
\usepackage{fullpage,amsthm,amsfonts,amssymb,epsfig,amsmath,times,amsthm}
\usepackage{algorithm,algpseudocode}
\usepackage{setspace}
\usepackage{tikz}

\newtheorem{theorem}{Theorem}
\newtheorem{claim}[theorem]{Claim}




\title{ CMPS 102 --- Quarter  Spring 2017 --  Homework 2}
\author{VLADOI MARIAN}
\date{\today}

\begin{document}
\maketitle



\begin{center}
{\bf I have read and agree to the collaboration policy.Vladoi Marian}\\
{\bf Name of students I worked with: Victor Shahbazian and Mitchell Etzel }
\end{center}


\section*{Solution to Problem 3: Greedy Algorithm,}

\textbf{3.a)} 

We have to consider the case : 
Mr Banks Period  of monitorization is (1, 15). \\
We have nurses covering : N1 = (1,4), N2=(4,15) , N3 = (2,14).   The algorithm in part a, will choose (N2, N3, N1) when the optimal solution should be (N1, N3).\\


\textbf{3.b)}  My greedy algorithm should take input a list of pairs of times $(s_i, f_i) $ for i = 1 to n. It also should take input Mr Banks Period  of monitorization ($s_p \  (start \ period), f_p \  (finish \ period))$\\ 
For each interval $k, (  1 \leq k \leq n )$ we create a maxHeap , were we insert all the intervals that overlaps $ k $ by finishing time as key. We will choose from this MaxHeap the interval which has the latest finishing time.
Algorithm that outputs a smallest subset of nurses that can cover the entire duration of Mr. Banks’ stay at the hospital or say that no such subset exists:\\

1. Sort all the intervals in ascending order by the starting time, so that $s_1 \leq s_2 ... \leq s_n$\\
2. Set A (that stores tha final intervals) = $\emptyset$\\
3. Choose the interval $x$ (which starts at $s_x== s_p$ and has the latest finishing time), and add it to A\\
4. While ($f_x \ does \ not \ equal \ f_p$ ) \\ \{ \\
5. If x maxHeap is empty.\\
6. Return no such subset  A exists.\\
7. From x maxHeap pop the interval  $k$  (this interval overlap x and  has the latest finishing time).\\
8. Add job k to A\\
9. x = k.\\
\} \\
10. Return the set A.\\


\textbf{3.c)}
The  algorithm return the  smallest subset of nurses that can cover the entire duration of Mr. Banks’ stay at the hospital or say that no such subset exist , and the algorithm is optimal.
\begin {proof}
We will proof this statement using greedy stays ahead strategy.\\
1. Suppose our gready algorithm is not optimal ( Proof by Contradiction).\\
2. Consider an optimal Solution. We will consider the optimal solution that agrees with gready solution for as many nurses as posible.\\
3.  Look at the first place where the optimal solution differs from the gready solution.\\
4. Let $ i_1, i_2, i_3 , i_k $ be the nurses intervals returned by  gready algorithm.\\
5. Let $ j_1, j_2, j_3,  j_ m $ be the nurses intervals returned by the  optimal algorithm.\\
6. For the largest value of r ,  $i_1 = j_1, i_2 = j_2,  ... i_r = j_r$ \\
7. If $r <  k$: .  It means that the optimal algorithm found at step (r+1) an interval that overlap the interval at step (r) and $f_{j_{r+1}} < f_{i_{r+1}}$, because our Gready Algorithm choses the interval that overlap with latest finish time.  In this case we can replace the $j_{r+1} \ with \  i_{r+1}$, and we get a contradiction because we get a optimal solution with larger r.\\
8. If r = k , but $m > k$, we get a contradiction because our algorith stopped at the end of Mr. Banks’ period of monitorization.  

\end {proof} 


\textbf{3.d)} Running time of the algorithm.\\
 Creating a max heap for each interval  $O (n log(b))$  where $b$  is the number of the elements in the heap. $b$  can not be greater that $n$  . This  $p$  Max heap has all the intervals that overlap the interval $p$  and return the one with the latest finish time.  I am not sure that I  can make this Max Heap  $O(nlog(n))$. \\
 a. step 1 Sorting the intervals  is $O nlog(n)$.\\
 b. step 2 and 3  of algorithm is $O (1).$\\
 c. step 4 to 9 $O (1).$\\
 Overall runnning time is O(nlog(n)).  O(n) space.



\end{document}
