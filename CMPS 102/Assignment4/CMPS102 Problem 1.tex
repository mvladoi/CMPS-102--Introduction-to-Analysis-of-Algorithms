\documentclass[11 pt]{article}
\usepackage{fullpage,amsthm,amsfonts,amssymb,epsfig,amsmath,times,amsthm}
\usepackage{algorithm,algpseudocode}

\newtheorem{theorem}{Theorem}
\newtheorem{claim}[theorem]{Claim}




\title{ CMPS 102 --- Quarter  Spring 2017 --  Homework 4}
\author{VLADOI MARIAN}
\date{\today}

\begin{document}
\maketitle

\begin{center}
{\bf I have read and agree to the collaboration policy.Vladoi Marian}\\
{\bf I want to choose homework heavy option.}\\
{\bf Name of students I worked with: Victor Shahbazian and Mitchell Etzel }
\end{center}


\section*{Solution to Problem 1 Node-capacitated networks : }

Assume we have a directed graph G = (V, E) with a source (s) and a destination (t). The node capacity for each node node $\in V $ , $c_v \geq 0$. The flow f in G through a node  $v$ is defined as $f^{in}(v) \leq  c_v$.

\textbf{a)} The algorithm  for the maximum flows in node-capacitated networks:\\

1. We create another Graph G' = (V', E').\\
2. The s $\in $ G becomes $s^{ou}t $ in G'. (Is the same node as s. It has all the edges that used to come out of s)\\
3. We link the node $s^{ou}t $ with all the nodes $v^{in}$. \\
4. The t $\in G$ becomes $t^{in}$ in G'. (Is the same node as t. It has all the edges that used to come into t)\\
5.  We link all the nodes $v^{out}$ with the node  $t^{in}$. \\
6. For all the other nodes $v \in G$, except s and t, we create two nodes $v^{in} \ and \  v^{out}$ in G'. ($v^{in} $ is the node $\in$ G' that has all the edges that used to come into node v $\in $ G. $v^{out}$ is 
the node $\in $ G' that has all the edges that used to come out of v $\in$ G).\\
7. For all the new edges in G' that we have created $c_e= \infty$.\\
8. The edge between the nodes $v^{in} \  and \ v^{out} $ has capacity $c_e = c_v  \ (v \in G)$. \\ 
9. We run Ford-Fulkerson described in class with capacity scaling on G' graph. \\
10. At the end of the algorithm we have a max flow from $ s^{out} \ to \ t^{in}$ in the G' . This is also the maximum flows in node-capacitated graph G.

To compute the max flow in node-capacitated graph G we have to compute the standard Ford - Fulkerson flow in the newly created graph G'.\\

\textbf{b,c)}
Assume there is a flow of value x in the original graph G. We can use this fact to construct a flow of the same value (x) in the new graph G'. The flow that passed through node v in G, in the new created graph G' will pass through the edge between  ( $v^{in} \ and \  v^{out}$). We know that this edge has capacity == $c_v , v \in G$. In the same time , if there is a flow of value (x) in G' it must be a flow of the same value (x) in G that respects all the constraints stated at the begining of the problem. The flow in G' was obtained by using the edges of the form ( $v^{in} , v^{out}$).

We consider the Max-Flow  Min-Cut theorem for node capacitated Graph G. We apply the same theorem on the modified graph G'. Assume we have any cut (A,B) $(s^{out} \in A \ and \ t^{in} B)$. The theorem states that the cut (A,B) has to be a minimum cut. The theorem also states that the max flow f = minimum capacity of the cut. This theorem was proved in class. Because we assigned capacity to be infinity to all the edges that come into $v^{in}$, and all the edges that come out $v^{out}$, we know that a min cut (A,B) will have only the edges between  ( $v^{in} \ and \  v^{out}$). Chosing any other edge that is not of the form ( $v^{in} , v^{out}$) will not create a min cut. Cutting all this edges will separate the source node and the sink node . In the original graph G , this cut can be interpreted as creating a set T of all this nodes $v$  that belong to the min cut (A,B) $\in$ G' , and deleting all the nodes in T. The capacity of the set T is the sum of all nodes capacities $\in$ T.  We are asked to define our "capacity" object . All the previous statements implies that the  value of maximal flow in a node capacitated network 
is equal to the minimum capacity of a set T that disconect the source node and the sink node. The correcteness of the Ford-Fulkerson was proved in class. \\

Because we use the Ford-Fulkerson with capacity scaling, the running time of our algorithm is O ($m^2 log C$) , where m is the number of edges in G' ,and C is the sum of all capacities edges that come out of the source node. 

\end{document}
