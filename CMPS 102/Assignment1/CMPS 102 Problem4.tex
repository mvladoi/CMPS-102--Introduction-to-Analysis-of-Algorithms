\documentclass[11pt]{article}
\usepackage{fullpage,amsthm,amsfonts,amssymb,epsfig,amsmath,times,amsthm}
\usepackage{algorithm,algpseudocode}
\usepackage{setspace}
\newtheorem{theorem}{Theorem}
\newtheorem{claim}[theorem]{Claim}




\title{ CMPS 102 --- Quarter  Spring 2017 --  Homework 1}
\author{VLADOI MARIAN}
\date{\today}

\begin{document}
\maketitle


\begin{center}
{\bf I have read and agree to the collaboration policy.Vladoi Marian}\\
{\bf I want to choose homework heavy option.}\\
{\bf Name of students I worked with: Victor Shahbazian and Mitchell Etzel }
\end{center}

\spacing{1.3}
\section*{Solution to Problem 4: Divide and Conquer}

One of the algorith that Jack and Anthony used to find the Median book of the joint book collection could be the folowing:
\begin{enumerate}
 \item We have 2 sets of sorted books , J and A, each set has the size n.
 We can find the median of this two sets in constant time. 
\item Find  median of J set , caled j ,   j= $J[ \left \lceil{\frac{n}{2}}\right \rceil  ]$  
 \item Find median of A set , caled a,  a = $A[ \left \lceil{\frac{n}{2}}\right \rceil  ]$ 
 \item  If $j == a$  return j (or a) , we found the median book 
 \item  If  $j > a$, then look for the median in the following subarrays : $J[ \left \lceil{\frac{n}{2}}\right \rceil  \ to \ (n-1) ]$ ,  $A[ 0 \ to \  \left \lceil{\frac{n}{2}}\right \rceil  ]$ 
 \item If  $j <  a$, then look for the median in the following subarrays : $J[ 0 \ to \ \left \lceil{\frac{n}{2}}\right \rceil]$ ,  $A[  \left \lceil{\frac{n}{2}}\right \rceil \ to \ (n-1) ]$ 
 \item Repeat the algorithm (divide and conquer) until size of both sets becomes 2
 \item If size of the two sets  is 2, we have 4 books or 3 books  . If $J[1] \leq  A[0] \ return \  J[1], \ else \    return  \  A[0] $.\\

 

  
\end{enumerate}

 Our subproblem size reduces by a factor of half and we spend only constant time to compare the    medians of J and A.The recurrence relation is : T(n) = T(n/2) + $\Theta(c)$ \\
Initial condition: the time to find the median book in a Set of size 2 is constant\\
T(1) = $\Theta$(1) = 1\\
Then we apply telescoping: reducing the recurrence relation to  n/2, n/4,  …, 2\\
T(n) = T($\frac{n}{2}$) + 1\\
T($\frac{n}{2}$) = T($\frac{n}{4}$) + 1\\
T($\frac{n}{4}$) = T($\frac{n}{8}$) + 1\\
 ...........................................\\
T(4) = T(2) + 1\\

Next we sum up the terms on left and right side of the ecuation:\\
T(n) + T($\frac{n}{2}$) + T($\frac{n}{4}$) + T($\frac{n}{8}$) ... + T(2) = T($\frac{n}{2}$) + T($\frac{n}{4}$) + T($\frac{n}{8}$) ... + T(2) + ( 1 + 1 + 1+ 1.. + 1)\\
There are (log n) number of 1's.\\
After canceling the terms on both side of the ecuation, we have \\
T(n) =  T(2) + (log n)\\
T(n) = c + log n\\
Therefore the runing time of the algorithm is T(n) = $\Theta$ (log n)\\
Proving by induction that the algorithm works.
\begin{claim}
Let P(n) be the assertion that algorithm works correctly for Sets of size n.Prove that P(n) is true for all n, then the algoritm works on all possible input Sets.
\end{claim}
\begin {proof}

\end {proof}
Base case: In the case where the size of the Sets is 2 , the algorith works.\\
Inductive Step: We assume that the algorithm works for Sets of size  $\leq k$ ;\\
We prove that the algorithm works for Sets of size k + 1;\\
When size of the sets is k+1 , we have 3 cases:\\
1. median of J == median of A(the algorithm works)\\
2. median of J$>$ median of A (in this case we look for the median on $J[ \left \lceil{\frac{n}{2}}\right \rceil  \ to \ (n-1) ]$ ,  $A[ 0 \ to \  \left \lceil{\frac{n}{2}}\right \rceil  ]$ ) So assuming that the recursive calls works correctly, this call works too.\\
3.median of J$<$ median of A (in this case we look for the median on $J[ 0 \ to \ \left \lceil{\frac{n}{2}}\right \rceil]$ ,  $A[  \left \lceil{\frac{n}{2}}\right \rceil \ to \ (n-1) ]$  ) So assuming that the recursive calls works correctly, this call works too.\\
The inductive step works correctly in all cases, we can conclude that the algorithm works correctly.

\end{document}
