\documentclass[10 pt]{article}
\usepackage{fullpage,amsthm,amsfonts,amssymb,epsfig,amsmath,times,amsthm}
\usepackage{algorithm,algpseudocode}

\newtheorem{theorem}{Theorem}
\newtheorem{claim}[theorem]{Claim}




\title{ CMPS 102 --- Quarter  Spring 2017 --  Homework 1}
\author{VLADOI MARIAN}
\date{\today}

\begin{document}
\maketitle

\begin{center}
{\bf I have read and agree to the collaboration policy.Vladoi Marian}\\
{\bf I want to choose homework heavy option.}\\
{\bf Name of students I worked with: Victor Shahbazian and Mitchell Etzel }
\end{center}


\section*{Solution to Problem 1: Resident Matcing}
 Google Matching Algorithm:
 We have n students applying for an intership at one of Google Team.
 We have m Google Teams , each with a fixed capacity, say p capacity.
 Each team had a ranking of the students in order of preference, and each student had a ranking of the teams in order of preference.
   Assume that there were more students who want an internship at Google than there were slots available in the  teams * capacity of the teams (m*p).We keep a min heap , representing the team , so we willbe able to add a new student in a team , or to remove a student from a team.
  The students do not have to play games. They rank the teams in order of preference even if they know that have no change to get to that team.
Ranking decision should be private.
The algorithm is both Students and Teams Friendly, more friendly to students because the core of the algorithm plays around the student' s preference list. 
 Matching a student and a team is called Tentative. A student might be tentatively matched with a team, and if he will be unmatched during the algorithm , an attempt is made to rematch the aplicant until his preference list will be exhausted. He might remain unmatched.
 At the beginig of the algorithm all students are free and all teams are empty. We keep all the free students in a Queue named Q, and all the matched students that are tentatively matched in Q complement $ \complement{Q} $. 


The pseudocode  for the Google Matching Algorithm:
\begin{algorithmic}
  \While { (Q with students is not empty) \\$And$ (The Student's rank ordered  list from Q is not exausted)}
    \State 
    Choose one student s  from the Q 
       \For{ (Check the rank ordered  List of students s )}
             \State 
                Choose first Team  t , from the ranked ordered List of students s 
                \If {(The Google Team t  has also ranked the Student  s)}
                     \State
                          \If {(The Google Team t  has an unfilled possition)}
                            \State Assign the Student  s and Google Team  t as a Tentative Match
                            \State Student  s is now in  $ \complement{Q} $
                              \ElsIf{(Another Student tentatively matched to that Team can be unmatched)}
                           \State
                          Un-assign the Student  x ranked lower by the Google Team
                          \State 
                          Push the un- assign Students x  from  $\complement{Q} $ back to  Q
                          \State
                         Assign the Google Team and  Student s  as tentative match.
                         \State Student  s is now in $\complement{Q} $
    \EndIf
     \EndIf
     \EndFor
     \EndWhile
    \ Return the set $\complement{Q} $ of matched teams and students.\\
  
\end{algorithmic}




The main ideas:\\
(1.1) Google teams are empty at the beginning of the algorithm. The teams have p capacity. The teams will accept all the tentative matching until it fill all the empty spots. It is guaranteed that the teams will fill the position because there are more students than teams positions. Google Teams, once it reached the capacity , remains with full capacity from the point at which it receives p+1 capacity proposal; and the sequence of students that it accept from that moment  gets better and better (in terms of its preference list).\\
(1.2) The sequence of teams to whom s apply gets worse and worse (in terms of his preference list).\\
(1.3) The running time of the Google Matching Algorithm $O((n+k )* m)$ 
Assuming $n$ students that apply for intership and $m$ Google Teams.
The outer $while$ loop will iterate at least for every student, and the inner $for$ loop will iterate at most m times ( this happens in the situation when all students were considered and they ranked all teams).
The running time can be calculated O(n*m).
Because we are talking about tentative matching, each time a student is unmatched, the outer $while$  loop has to repeat for the student unmatched. Consider there are $k$ tentative matching canceled (to remove or to add a student is log p ) the running time will be $O((n+k )* m)$  which can be considered quadratic time performance $O(n^2)$. For the Space complexity we have n*m matrix, m*n matrix , and each team has a priority queue data structure. Considering that the capaciy of all teams can not be bigger that number of students, ($pm <= n, m =  \frac{n}{p}$), the total space complexity is $f(n) = nm + n = n( m+1) =  n(\frac{n}{p} + 1) =  \frac{n^2}{p} + n = O(n^2). $  
(1.4) If s is free at the end of this algorithm,then it means that all Google Teams ranked highter other applicants than him.\\
(1.5) The set $\complement{Q} $ returned at termination is a perfect matching. This means that the teams have chosen the best aplicants considering their's ordered ranking List. Let us suppose that the algorithm terminates with a free students s that is ranked higher in at least one of Google Team. At termination of outer while loop, it must be the case that s had already proposed to each team from his ordered ranking list. So if the Google team also ranked him higher, then will be accepted, which is a contradiction.\\
(1.6) Consider an execution of the Google Matching  algorithm that returns a set of pairs  $\complement{Q} $. The set $\complement{Q} $ is a stable matching.We have already shown in (1.5)  that $\complement{Q} $  is a perfect matching. We will proove this by contradiction. Assume that is an instability in $\complement{Q} $.\\
This type of instability that can occur is that there are teams $t_0$ and $t_1$ and students $s_0$ and $s_1$ so that: \\
- $s_0$ is matched with $t_0$, and  $s_1$ is matched with $t_1$, and \\
- $t_0$ favors $s_1$ over $s_0$, and  $s_0$ favors $t_0$  over $t_1$. 

In the execution of the algorithm that produced $\complement{Q} $,  $s_0'$s last application was, by definition, to  $t_0$. Then we ask : did $s_0$ apply to $t_1$ at some earlier point in this execution? If he didn’t, then $t_0$  must occur higher on $s_0$'s preference list than $t_1$, contradicting our assumption that $s_0$ prefers $t_1$ to $t_0$ . If he did, then he was rejected by $t_1$  in favor of some other student $s_2$, whom $t_1$ prefers to $s_0$ . $s_1$ is one of the student accepted by  $t_1$  so either $s_2 == s_1$ or  $t_1$ prefers its  final student  $s_1$ to $s_2$  either way this contradicts our assumption that $t_1$  prefers $s_0$ to $s_1$ .
It follows that $\complement{Q} $, is a stable matching.





\end{document}
